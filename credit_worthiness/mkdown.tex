\documentclass[]{article}
\usepackage{lmodern}
\usepackage{amssymb,amsmath}
\usepackage{ifxetex,ifluatex}
\usepackage{fixltx2e} % provides \textsubscript
\ifnum 0\ifxetex 1\fi\ifluatex 1\fi=0 % if pdftex
  \usepackage[T1]{fontenc}
  \usepackage[utf8]{inputenc}
\else % if luatex or xelatex
  \ifxetex
    \usepackage{mathspec}
  \else
    \usepackage{fontspec}
  \fi
  \defaultfontfeatures{Ligatures=TeX,Scale=MatchLowercase}
\fi
% use upquote if available, for straight quotes in verbatim environments
\IfFileExists{upquote.sty}{\usepackage{upquote}}{}
% use microtype if available
\IfFileExists{microtype.sty}{%
\usepackage{microtype}
\UseMicrotypeSet[protrusion]{basicmath} % disable protrusion for tt fonts
}{}
\usepackage[margin=1in]{geometry}
\usepackage{hyperref}
\hypersetup{unicode=true,
            pdftitle={Untitled},
            pdfauthor={Niall Trinder - R00088254},
            pdfborder={0 0 0},
            breaklinks=true}
\urlstyle{same}  % don't use monospace font for urls
\usepackage{graphicx,grffile}
\makeatletter
\def\maxwidth{\ifdim\Gin@nat@width>\linewidth\linewidth\else\Gin@nat@width\fi}
\def\maxheight{\ifdim\Gin@nat@height>\textheight\textheight\else\Gin@nat@height\fi}
\makeatother
% Scale images if necessary, so that they will not overflow the page
% margins by default, and it is still possible to overwrite the defaults
% using explicit options in \includegraphics[width, height, ...]{}
\setkeys{Gin}{width=\maxwidth,height=\maxheight,keepaspectratio}
\IfFileExists{parskip.sty}{%
\usepackage{parskip}
}{% else
\setlength{\parindent}{0pt}
\setlength{\parskip}{6pt plus 2pt minus 1pt}
}
\setlength{\emergencystretch}{3em}  % prevent overfull lines
\providecommand{\tightlist}{%
  \setlength{\itemsep}{0pt}\setlength{\parskip}{0pt}}
\setcounter{secnumdepth}{0}
% Redefines (sub)paragraphs to behave more like sections
\ifx\paragraph\undefined\else
\let\oldparagraph\paragraph
\renewcommand{\paragraph}[1]{\oldparagraph{#1}\mbox{}}
\fi
\ifx\subparagraph\undefined\else
\let\oldsubparagraph\subparagraph
\renewcommand{\subparagraph}[1]{\oldsubparagraph{#1}\mbox{}}
\fi

%%% Use protect on footnotes to avoid problems with footnotes in titles
\let\rmarkdownfootnote\footnote%
\def\footnote{\protect\rmarkdownfootnote}

%%% Change title format to be more compact
\usepackage{titling}

% Create subtitle command for use in maketitle
\newcommand{\subtitle}[1]{
  \posttitle{
    \begin{center}\large#1\end{center}
    }
}

\setlength{\droptitle}{-2em}

  \title{Untitled}
    \pretitle{\vspace{\droptitle}\centering\huge}
  \posttitle{\par}
    \author{Niall Trinder - R00088254}
    \preauthor{\centering\large\emph}
  \postauthor{\par}
      \predate{\centering\large\emph}
  \postdate{\par}
    \date{3 December 2018}


\begin{document}
\maketitle

\begin{center}\rule{0.5\linewidth}{\linethickness}\end{center}

\subsection{Introduction}\label{introduction}

For this assignment, we're working on the premise that a manager from a
local financial institution, Olivia, has contacted us to help her assess
the credit worthiness of future potential customers.

She has provided us with the following data:

\begin{itemize}
\tightlist
\item
  Data set 1; 793 past loan customers with 14 attributes, labelled based
  on Good/Bad Credit Standing.
\item
  Data set 2; 10 potential loan customers with no labels.
\end{itemize}

Our goal is to explore and assess the past loan customer data set, chose
an appropriate model to train using this data and then use that model to
assess the potential customer data set to aid Olivia's financial
institution minimize the risk of loaning money to a customer who may
default on the loan.

\subsection{Exploratory Data Analysis
(EDA)}\label{exploratory-data-analysis-eda}

\subsection{ROC Curves (\textasciitilde{}280
words)}\label{roc-curves-280-words}

A ROC curve is a plot of the ``true positive rate'' on the y-axis and
the ``false positive rate'' on the x-axis for every possible
classification threshold.

The true positive rate (or sensitivity) is the number of true positive
classifications divided by the total occurances of the true class. For
instance; if you were to classify a patient has `having the disease' or
`not having the disease' then the true positive rate would be the number
of patients \emph{correctly} classified as having the disease divided by
the true total number of patients who have the disease.

The false positive rate (or specificity) is simply 1 - sensitivity, or,
the number of false positive classifications divided by the total
occurances of the false class.

The ROC courve visualises all possible thresholds were as
missclassification rate is the error rate for a single threshold.

A diagonal line from (0, 0) to (1, 1) on the graph would represent a
model that does no better than guessing.

You can use the ROC curve to quantify the performance of the classifier
by giving a higher rating to better performing models. To do this we
evaluate the area under the curve and express it as a percentage of the
total area.

Note that most problems in the real world don't have balanced classes
and that this does not affect the ROC curve. Also ROC curves are useful
een if the predicted values are not properly calibrated.


\end{document}
